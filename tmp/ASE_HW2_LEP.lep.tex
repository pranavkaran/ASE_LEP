\title{CS 7140 Advanced Software Engineering \\ Homework 2 Question 1}
\author{Prof: Dr. Mateti \\ Submitted by: Pranav Pranav}
\date{26 July 2016}

\documentclass{article} \usepackage{listings} \lstset{numbers=left,
numberstyle=\tiny,language=Java,basicstyle=\small,
tabsize=2,breaklines,showstringspaces=false,frame=tB} \lstset{literate=
{ö}{{\"o}}1 {ä}{{\"a}}1 {ü}{{\"u}}1 {Ö}{{\"O}}1 {Ü}{{\"U}}1 {Ä}{{\"A}}1
{ß}{{\ss{}}}1} \usepackage[T1]{fontenc}
\usepackage[utf8]{inputenc} \usepackage{mathpazo}
\usepackage[linktocpage=true,pdftex=true, colorlinks=false]{hyperref} \sloppy

\begin{document}

\maketitle

\section{Introduction}
This class implements a set of integers. It is limited by the size sz chosen at
compile time. This is expected to be a good example of software engineering in
the tiny. Possibly contains bugs. We naively assume all objects are non-null.
It is also intended as an example for use with learning JUnit/TestNG, JML and
FindBugs.In this document we will go through the implementation of a small set
class implemented in java by Dr. Mateti.
For further details, see
http://cecs.wright.edu/~pmateti/Courses/7140/Lectures
\footnote{http://cecs.wright.edu/~pmateti/Courses/7140/Lectures}
/Design/smallSet-DesignDoc.html; long names have been shortened.
\footnote{/Design/smallSet-DesignDoc.html}


\subsection{Small Set Class}
Apparantly javadoc has no tags for pre- and post-conditions; so
I am placing assert()s at the top of the methods for pre- and at
the bottom for post.  As is, we are not using JML syntax.  The old
objects, setOf() and classInv() are needed only because of asserts.
The obj1 == obj2 used below is intended to be deep equality.

The class SMALLSET contains three properties (ear, ux, sz). The elements are
stored in (ear) array while (ux) is an index that is always referring to the
position after the last element in the set. The property (sz) is the maximam
size of the set. The set is initialized using the constructors as shown in the
code below. To maintain sets the class SMALLSET exploit these functions:
\begin{itemize}
    \item SmallSet setOf(SmallSet s)
    \item SmallSet setOf(int [] a, int m, int n)
    \item indexOf(int e)
    \item isin(int e)
    \item SmallSet compact()
    \item SmallSet insert(int e)
    \item delete(int a, int b, int e)
    \item delete(int e)
    \item int nitems()
    \item SmallSet union(SmallSet tba)
    \item SmallSet diff(SmallSet tbs)
    \item String toString()
    \item main(String[] args)
\end{itemize}
The upcoming sections explain these functions in more details.
\begin{lstlisting}[title={</SmallSet.java 1>}, label=Listing1]
public class SmallSet {
    private int[] ear;         
    private int ux = 0;        
    private int sz = 1000;
    public SmallSet() {
        ear = new int[sz];
    }

    public SmallSet(int [] a, int m, int n) {
        assert m < n;
        ear = new int[sz];
        for (int i = m; i < n; i++) {
            insert(a[i]);
        }
    }
    <<classinvariant 2>>
    <<set Of 3>>
    <<indexOf 4>>
    <<isin 5>>
    <<Compact 6>>
    <<Insert 7>>
    <<Delete 8>>
	<<nitems 9>>
	<<Union and Diff 10>>
	<<ToString 11>>
	<<main 12>>
}
\end{lstlisting}\begin{footnotesize} \textsc{Included Blocks}: \hyperref[Listing2]{2 on page} \pageref{Listing2}, \hyperref[Listing3]{3 on page} \pageref{Listing3}, \hyperref[Listing4]{4 on page} \pageref{Listing4}, \hyperref[Listing5]{5 on page} \pageref{Listing5}, \hyperref[Listing6]{6 on page} \pageref{Listing6}, \hyperref[Listing7]{7 on page} \pageref{Listing7}, \hyperref[Listing8]{8 on page} \pageref{Listing8}, \hyperref[Listing9]{9 on page} \pageref{Listing9}, \hyperref[Listing10]{1\-0 on page} \pageref{Listing10}, \hyperref[Listing11]{1\-1 on page} \pageref{Listing11}, \hyperref[Listing12]{1\-2 on page} \pageref{Listing12}\end{footnotesize}\vskip 5mm\noindent
\subsubsection{Class Invariant}
As to make sure of the correctness of the object, the objects must always holds
true. This can be achieved by making sure the index (ux) is not less than 0 and
not more than size (sz). This function is needed only for assert.
@param s any SmallSet
Class invariant, expressed as a boolean function.
@return truthhood of the boolen exp of the invariant
\begin{lstlisting}[title={<classinvariant 2>}, label=Listing2]
public boolean classInv(SmallSet s) {
        return
            0 <= s.ux && s.ux < s.sz
            && setOf(s) == setOf(s.ear, 0, ux-1)
            ;
    }
\end{lstlisting}\begin{footnotesize} \textsc{Used in}: \hyperref[Listing1]{/\-S\-m\-a\-l\-l\-S\-e\-t\-.\-j\-a\-v\-a on page} \pageref{Listing1}  \end{footnotesize}\vskip 5mm\noindent
\subsubsection{(Set Of)}
The setOf() function is needed only for assert.
\begin{lstlisting}[title={<set Of 3>}, label=Listing3]
private SmallSet setOf(SmallSet s) {
        return s.compact();
    }

    private SmallSet setOf(int [] a, int m, int n) {
        return setOf(new SmallSet(a, m, n));
    }
\end{lstlisting}\begin{footnotesize} \textsc{Used in}: \hyperref[Listing1]{/\-S\-m\-a\-l\-l\-S\-e\-t\-.\-j\-a\-v\-a on page} \pageref{Listing1}  \end{footnotesize}\vskip 5mm\noindent

\subsubsection{Returining the index of an element}
Returning the index of an element (e).
\begin{lstlisting}[title={<indexOf 4>}, label=Listing4]

    private int indexOf(int e) {
        assert classInv(this);
        int i = 0;
        ear[ux] = e;
        while (ear[i] != e)
            i++;
        assert 0 <= i && i <= ux && e == ear[i] ;
        return i;
    }
\end{lstlisting}\begin{footnotesize} \textsc{Used in}: \hyperref[Listing1]{/\-S\-m\-a\-l\-l\-S\-e\-t\-.\-j\-a\-v\-a on page} \pageref{Listing1}  \end{footnotesize}\vskip 5mm\noindent
\subsubsection{Checking if the element exist}
This function return true if element (e) exist otherwise it returns false. This
function invoking indexOf(e) to return the index of an element.
@param e element to search for
@return any i such that ear[i] == e
\begin{lstlisting}[title={<isin 5>}, label=Listing5]
 public boolean isin(int e) {
        int x = indexOf(e);
        assert x >= ux || (x == ear[x] && 0 <= x && x < ux);
        return (x < ux);
    } 
\end{lstlisting}\begin{footnotesize} \textsc{Used in}: \hyperref[Listing1]{/\-S\-m\-a\-l\-l\-S\-e\-t\-.\-j\-a\-v\-a on page} \pageref{Listing1}  \end{footnotesize}\vskip 5mm\noindent
\subsubsection{Compact}
Compacting the elements to avoid reputation of an element by deleting elements
one by one and copying it to the new object (newset) and returning it. While
keeping the abstraction intact, compactify the ear[] so that ux can be the
lowest possible.Rewrite it without using newset.
@return the new SmallSet equal to setOf(this)
\begin{lstlisting}[title={<Compact 6>}, label=Listing6]
  public SmallSet compact() {
        SmallSet newset = new SmallSet();
        while (ux > 0) {
            int e = ear[ux-1];
            delete(e);
            if (! newset.isin(e))
                newset.insert(e);
        }
        assert setOf(this) == setOf(newset) && newset.ux <= this.ux;
        return newset;
    }
\end{lstlisting}\begin{footnotesize} \textsc{Used in}: \hyperref[Listing1]{/\-S\-m\-a\-l\-l\-S\-e\-t\-.\-j\-a\-v\-a on page} \pageref{Listing1}  \end{footnotesize}\vskip 5mm\noindent
\subsubsection{Insert}
Inserting an element is done by copying the first element to where (ux) is
pointing and then replace the first element with the element that need to be inserted.

\begin{lstlisting}[title={<Insert 7>}, label=Listing7]
    public SmallSet insert(int e) {
        if (ux < sz - 1) {
            if (ux > 0)
                ear[ux] = ear[0];
            ear[0] = e;
            ux++;
        } else {

        }
        assert ux == sz - 1 || isin(e);
        return this;
    }
\end{lstlisting}\begin{footnotesize} \textsc{Used in}: \hyperref[Listing1]{/\-S\-m\-a\-l\-l\-S\-e\-t\-.\-j\-a\-v\-a on page} \pageref{Listing1}  \end{footnotesize}\vskip 5mm\noindent
\subsubsection{Delete}
To delete element (e) the function with one parameter is first invoked and then
in turn invoking the delete function with the three parameters. where (a) is 0,
(b) is the index. This function is returning the number of required deletion
(nd).
Delete all occurrences of e in ear[a to b-1], and compact ear[].
A casual reader comments: This is tricky!  Do write a loop invariant.
@return the count of deletions.
\begin{lstlisting}[title={<Delete 8>}, label=Listing8]
    private int delete(int a, int b, int e) {
        assert 0 <= a && a < b;
        int nd = 0;

        for (int i = a, j = a; i < b; i++) {
            if (ear[i] == e)
                nd ++;
            else {
                if (j < i)
                    ear[j] = ear[i];
                j++;
            }
        }
        assert 0 <= nd && nd < b - a && ! setOf(ear, a, b).isin(e);
        return nd;
    }
    

    public SmallSet delete(int e) {
        assert ux < sz;
        ux -= delete(0, ux, e);
        assert ux < sz && !isin(e);
        return this;
    }
\end{lstlisting}\begin{footnotesize} \textsc{Used in}: \hyperref[Listing1]{/\-S\-m\-a\-l\-l\-S\-e\-t\-.\-j\-a\-v\-a on page} \pageref{Listing1}  \end{footnotesize}\vskip 5mm\noindent
\subsubsection{Returning number of items}
Returning the number of items in the set that is represented by the index (ux).
Invoking the compact() is necessary to compact the set and avoid reputation.
\begin{lstlisting}[title={<nitems 9>}, label=Listing9]
    public int nitems() {
        SmallSet s = compact();
        ux = s.ux;
        ear = s.ear;
        assert ux == setOf(s).nitems();
        return ux;
    }
\end{lstlisting}\begin{footnotesize} \textsc{Used in}: \hyperref[Listing1]{/\-S\-m\-a\-l\-l\-S\-e\-t\-.\-j\-a\-v\-a on page} \pageref{Listing1}  \end{footnotesize}\vskip 5mm\noindent
\subsubsection{Adding and Subtracting Sets}
Concatenating two sets is done by the union() function where (tba) represent
the set to be added. Subtracting two sets is doen by the diff() function where
(tbs) represent the set that need to be subtracted.

\begin{lstlisting}[title={<Union and Diff 10>}, label=Listing10]
   public SmallSet union(SmallSet tba) {
        SmallSet old = this;
        for (int i = 0; i < tba.ux; i++)
            insert(tba.ear[i]);
        assert setOf(this) == setOf(old).union(setOf(tba));
        return this;
    }

    public SmallSet diff(SmallSet tbs) {
        SmallSet old = this;
        SmallSet newset = new SmallSet();
        for (int i = 0; i < tbs.ux; i++)
            if (! this.isin(tbs.ear[i])) newset.insert(tbs.ear[i]);
        for (int i = 0; i < this.ux; i++) {
            if (! tbs.isin(this.ear[i])) newset.insert(this.ear[i]);
        }
        ear = newset.ear;
        ux = newset.ux;
        assert setOf(this) == setOf(old).diff(setOf(tbs));
        return this;
    }
\end{lstlisting}\begin{footnotesize} \textsc{Used in}: \hyperref[Listing1]{/\-S\-m\-a\-l\-l\-S\-e\-t\-.\-j\-a\-v\-a on page} \pageref{Listing1}  \end{footnotesize}\vskip 5mm\noindent
\subsubsection{Printing nicely organized set}
This function override the original ToString() to provide a customized printing
pattern for the set.
 
\begin{lstlisting}[title={<ToString 11>}, label=Listing11]
  public String toString() {
        String s = "el: ";
        for (int i = 0; i < ux; i++) {
            s += ", " + ear[i];
        }
        s += ", ux=" + ux;
        return s;
    }
\end{lstlisting}\begin{footnotesize} \textsc{Used in}: \hyperref[Listing1]{/\-S\-m\-a\-l\-l\-S\-e\-t\-.\-j\-a\-v\-a on page} \pageref{Listing1}  \end{footnotesize}\vskip 5mm\noindent
\subsubsection{Testing some data}
Some test data to test the implemented class SMALLSET.
\begin{lstlisting}[title={<main 12>}, label=Listing12]
  public static void main(String[] args) {
        SmallSet s = new SmallSet();
        SmallSet t = new SmallSet();
        int [] a = {1,2,3,4,5,6};
        for (int j = 0; j < 3; j++)
            for (int i = 0; i < a.length; i++) { // or use setOf()

                s.insert(a[i]);
                t.insert(a[i] - 1);
            }
        // some simple tests

        System.out.println("set s " + s + "; nitems=" + s.nitems());
        s.delete(1);
        s.delete(3);
        System.out.println("set s " + s + "; nitems=" + s.nitems());
        s.union(t);
        System.out.println("set s " + s + "; nitems=" + s.nitems());
        t.insert(7);
        t.diff(s);
        System.out.println("set t " + t + "; nitems=" + t.nitems());
    }
\end{lstlisting}\begin{footnotesize} \textsc{Used in}: \hyperref[Listing1]{/\-S\-m\-a\-l\-l\-S\-e\-t\-.\-j\-a\-v\-a on page} \pageref{Listing1}  \end{footnotesize}\vskip 5mm\noindent

\subsubsection{The output}
set s el: , 6, 1, 2, 3, 4, 5, 6, 1, 2, 3, 4, 5, 6, 1, 2, 3, 4, 5, ux=18; nitems=6
set s el: , 6, 5, 4, 2, ux=4; nitems=4
set s el: , 4, 2, 4, 5, 6, 5, 0, 1, 2, 3, 4, 5, 0, 1, 2, 3, 4, 5, 0, 1, 2, 3, ux=22; nitems=7
set t el: , 7, 6, ux=2; nitems=2
\section{Discussion}
This program is designed in a way that meet the Design by Contract. The class
invariant, post-conditions, and preconditions are used rigorously in this
class. One issue and as shown in the output section the set when printed for the first time it is not
compacted, thus, there is a need to invoke nitems() before printing the set


\end{document}